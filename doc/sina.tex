\documentclass[english,a4paper]{article}
\usepackage[latin1]{inputenc}
\usepackage{verbatim}
\usepackage[bookmarksopen,bookmarksnumbered]{hyperref}
\usepackage[fancyhdr]{latex2man}
%\usepackage{svninfo}
%\setDate{\svnInfoLongDate}
\setDate{May 14th, 2014}
\setVersion{1.4.0-rc}
%\setVersionWord{v}

%\makeatletter
%\renewcommand{\@LM@Opt}[1]{\textbf{#1}}
%\makeatother
\renewcommand{\URL}[1]{\url{#1}}
\setcounter{tocdepth}{1}
\begin{document}
\begin{Name}{1}{sina}{Elmar Pruesse}{Multiple Sequence Alignment}{SINA: high throughput multiple sequence alignment}
SINA is a tool for aligning sequences with an existing multiple sequence alignment (MSA) at high accuracy. It can also execute a homology search based
on the computed alignment and generate a per sequence classifications from the search results. 

This manual documents the command line usage of sina. Please see \URL{http://www.arb-silva.de/aligner} for a reference 
to the scientific description of the employed algorithms.

\end{Name}
%@% IF LATEX %@%
\vfill
\tableofcontents
\vfill
\newpage
%@% END-IF %@%


\section{Synopsis}
\Prog{sina} \OptArg{-i }{sequences}.{\texttt{fasta}\Bar\texttt{arb}} \OptArg{-o }{output}.{\texttt{fasta}\Bar\texttt{arb}} 
\{\Opt{-{}-prealigned} \Bar\ \Opt{ -{}-ptdb} \Arg{aligndb}.\texttt{arb}\} [\Opt{-{}-search} \OptArg{-{}-search-db }{searchdb}\texttt{.arb}]  \oArg{options}
\section{Description}

\noindent You can view SINA as a one-command pipeline composed of the following stages:

\begin{enumerate}
\item Read sequences from FASTA or ARB file.
\item Align sequences with reference MSA.
\item Search for most similar sequences in search MSA.
\item Classify sequences using search result.
\item Write sequences to FASTA or ARB file.
\end{enumerate}

You can enable or disable the middle three stages as required. By default, only the alignment stage is
enabled. Briefly, this is what those stages do (see section Options for details on the configuration
options accepted by each of the stages). 

\begin{Description}[Classify:]
\item[Read:    ] Reads sequences from a multi-FASTA file or an ARB database. If reading from ARB, additional 
meta-data can be read as key-value pairs. These key-value pairs are carried with each sequence throughout 
the pipeline and will be exported at the end. 
\item[Align:] Sequences are aligned using the POA algorithm with a set of reference sequences drawn from the reference 
MSA. Reference sequence selection is based on a kmer search.
\item[Search:] Sequences are compared after alignment with the aligned sequences in the configured search database.
Comparison is done either against all sequences or against the best matches from a kmer search. Identity is 
computed as the number of identical column/base pairs divided by the length of the query sequence.
\item[Classify:] Sequence classification uses least-common-ancestor (LCA) to derive a classification from the
classifications of the sequences found during the search stage. 
\item[Write:] Writes sequences and meta-data to a multi-FASTA file or an ARB database. See section Options for 
possible format options to export meta data when writing to multi-FASTA.
\end{Description}

The default parameters are a pretty good starting point. They were optimized using a large SSU rRNA gene reference
MSA. If you want to use SINA for other gene sequences, see section Examples on how to do some simple 
accuracy benchmarks on them. To improve the results, the parameters you will want to start with are 
\Opt{--fs-full-len} (set to the typical size of a full-length sequence) and \Opt{--fs-kmer-len} (setting
this to 8 may help with more variable or shorter sequences). 

\section{Options}
Options beginning with a single ``-'' must be separated from arguments with a space character. Options beginning 
with ``-{}-'' can also be separated from arguments with an equal sign. 
\subsection{General Options}
\begin{Description}
\item[\Opt{-h}, \Opt{-{}-help}] 
  Print a summary of the available options and exit.
\item[\Opt{-{}-version}] 
  Print the version information and exit.
\item[\Opt{-{}-show-conf}] 
  Print a summary of all configuration settings before processing the input sequences.
\item[\OptArg{-i }{filename}, \OptArg{-{}-in=}{filename}] 
  Specify the source file containing the sequences to be aligned. The special filename ``:'' can be used to 
  access an open ARB database when starting SINA from a shell spawned from within ARB. The sequence data
  may already be aligned (and should be, if you supply \Arg{-{}-prealigned}).
\item[\Opt{-{}-intype} \{\Arg{fasta}\Bar\Arg{arb}\}]
  Specify the format of the source file. If the filename ends with ``arb'' or ``fasta'', the type 
  is automatically deduced. 
\item[\OptArg{-o }{filename}, \OptArg{-{}-out=}{filename}]
  Specify the destination file for the aligned sequences. The special filename ``:'' can be used to access 
  an open ARB database when starting SINA from a shell spawned from within ARB. If you want to discard 
  the aligned sequences, you can set \Arg{filename} to \Arg{/dev/null} and \Opt{-{}-outtype} to \Arg{fasta}.
\item[\OptArg{-{}-outtype }{fasta\Bar arb}]
  Specify the format of the destination file. If the filename ends with ``arb'' or ``fasta'', the type 
  is automatically deduced. 
\item[\Opt{-{}-prealigned}]
  If set, the alignment stage is disabled. Sequences are passed to search (if enabled) and output
  stage unmodified. Mandatory alignment parameters (\Opt{-{}-ptdb}) are not required in this case.
  The input file should contain correctly aligned sequences. 
\item[\Opt{-{}-search}]
  If set, the search stage is enabled. 
\end{Description}
\subsection{Logging Options}
\begin{Description}
\item[\Opt{-{}-show-diff}] 
This flag enables visualization of alignment differences. This feature allows you to 
quickly assess where your alignment differs from the one SINA computed. By also showing you the alignment of 
the reference sequences used for aligning the sequence, you can get an idea of why SINA came to its conclusions.
Many cases of ``suboptimal'' alignment can be attributed to inconsistent alignment of the reference sequences. 
To fix such problems, you could either correct the alignment of the reference sequences or add your corrected
sequence to the reference alignment. 

Alignment difference visualization requires that the input sequences be 
already aligned in a way compatible with the used reference alignment. For positions at which the original alignment and
the alignment computed by SINA differ, output as shown below will be printed to the log:
\begin{verbatim}

   Dumping pos 1121 through 1141:
   -{}--{}--{}--{}--  4 14 16-17 21 24
   G-C-AGUC-  40 <-{}--(%% ORIG %%)
   GCA-{}-GUC-  41 <-{}--(## NEW ##)
   GCA-AGUC-  0-3 5-13 15 18-20 22-23 25-27 29-39
   GCAA-GUC-  28
\end{verbatim}

In this case, the bases '\texttt{C}' and '\texttt{A}' where placed in other columns than as per the original alignment. The original alignment
is marked with \texttt{<-{}--(\%\% ORIG \%\%)}. The new alignment is marked with \texttt{<-{}--(\#\# NEW \#\#)}. The numbers 
to the right of the alignment excerpt indicate the indices of the sequences in the alignment reference (field 
\Arg{align\_family\_slv}) which the respective row represents. All-gap columns are not shown. The first line indicates
the range of alignment columns displayed. 
\item[\Opt{-{}-show-dist}]
This flag enables computing the values \Arg{sps}, \Arg{error}, \Arg{matches}, \Arg{mismatches}, \Arg{bps}, \Arg{cpm}, \Arg{idty} and \Arg{achieved\_idty}. See section ``Generated Meta Data Values'' for an explanation of the individual values. All values except \Arg{bps} are computed by comparing the newly computed alignment with the original alignment of the sequences. If a database is configured using \Opt{-{}-orig-db}, the original alignment is obtained from that database. Otherwise, the alignment of the input sequences is used. 
\item[\OptArg{-{}-orig-db }{arb database}]
The database \Arg{arb database} is used to retrieve a aligned sequences to be used as a reference for comparison by \Opt{-{}-show-diff} and \Opt{-{}-show-dist}. Sequences are retrieved based on the contents of the ARB field \Arg{name}. If FASTA is used as input format, the first word of the FASTA header will be used for matching. 
\item[\Opt{-{}-colors}]
Enable color in the output of \Opt{-{}-show-diff}.
\item[\OptArg{-{}-log-file }{filename}]
Redirect the log output to \Arg{filename}. 
\end{Description}
\subsection{Reading from ARB}
\begin{Description}
%\item[\Opt{-{}-markcopied}]
%\item[\Opt{-{}-markaligned}]
\item[\OptArg{-{}-select-file }{filename}]
If using an ARB database as sequence input file, only sequences with a \Arg{name} matching a line contained
within \Arg{filename} will be passed into alignment and search stages. 
\item[\OptArg{-{}-select-step }{n}]
If using an ARB database as sequence input file, only every \Arg{n}th sequence will 
be passed into alignment and search stages. This may be combined with \Arg{-{}-select-file}.
In combination with \Opt{-{}-select-skip} this option can be used
transparently distribute processing of a single ARB database to multiple instances of SINA. 
\item[\OptArg{-{}-select-skip }{n}]
If using an ARB database as sequence input file, the first \Arg{n} sequences will be skipped. Combination
with \Arg{-{}-select-file} is possible. In combination with \Opt{-{}-select-step} this option can be used
transparently distribute processing of a single ARB database to multiple instances of SINA
\item[\OptArg{-{}-extra-fields }{fieldnames}]
Passing a colon separated list of field names will load the meta data contained within these database fields
from ARB. The contents will be passed as key-value pairs through the internal SINA pipeline. They will 
be treated like meta data generated by SINA itself. That is, they will be printed to the log file and
written to the output file. 
\end{Description}
\subsection{Writing to ARB}
\begin{Description}
\item[\OptArg{-{}-prot-level }{n}]
Set the protection level used to write to the ARB database to \Arg{n}. If a field was set to have a protection 
level above \Arg{n}, SINA will (silently) fail to write to these fields. If your sequences have a protection 
level of for example 4 and you set \Arg{n} to 0, your sequence data will not be modified. If you use the 
same ARB database for input and output, this may be used in combination with \Opt{-{}-show-diff} to inspect
the effect of varying the alignment parameters without modifying the alignment.
\end{Description}
\subsection{Writing to FASTA}
\begin{Description}
\item[\Opt{-{}-meta-fmt }\{\Arg{none}\Bar\Arg{header}\Bar\Arg{comment}\Bar\Arg{csv}\}] 
This option configures the 
format in which meta data will be exported if the output format is FASTA. \Arg{none} will discard all meta data (it 
will still be written to the log, however). \Arg{header} will export meta data values as bracket enclosed key
value pairs on the FASTA header line. \Arg{comment} will export meta data values as key value pairs on FASTA 
comment lines, that is lines beginning with a semicolon between the header and the sequence data. \Arg{csv} will 
export meta data values to a separate file in RFC4180 compatible comma separated value format. The filename
will be generated from the output filename by appending ``\texttt{.csv} ''. 
%\item[\Opt{-{}-copy-ref}]
%\item[\OptArg{-{}-copylimit}]
\item[\OptArg{-{}-line-length }{n}] 
If \Arg{n} is different from 0, sequence data will be line wrapped after \Arg{n}
characters. 
\end{Description}
\subsection{Alignment Options}
\begin{Description}
\item[\OptArg{-{}-ptdb }{filename}] 
Specifies the ARB database to be used as alignment reference. This is a mandatory
parameter. The file must be in ARB format. See section Examples below for an explanation how to generate such a
database from a FASTA file using only SINA. The name of this parameter is historical and refers to the
fact that a ARB PT server will be started using the configured database to search for the sequences having 
the least kmer distance to the input sequences.
\item[\OptArg{-{}-ptport }{socket}] 
Configures the socket which will be used for communication with the ARB PT server. SINA will attempt to 
contact a running PT server via this port. If no PT server can be contacted, SINA will attempt to start
one itself. 

\Arg{socket} may either be of the format \Arg{hostname:port}, specifying a TCP socket, or of the format
\Arg{:filename}, specifying a Unix socket. If no running PT server could be contacted and a Unix socket 
is specified or hostname is ``\texttt{localhost}'', a PT server will be started locally. If \Arg{hostname} is ``\texttt{\_\_SGE\_\_}''
SINA will start and contact a PT server on a cluster node using \Cmd{qrsh}{1}. Otherwise, \Cmd{ssh}{1} 
will be used to start a PT server on the configured host. 

The default is to use port ``\texttt{localhost:4040}''.

CAUTION: If a PT server is already running on the configured socket, but its database does not match the 
database configured with \Opt{-{}-ptdb} the results will be undefined. The search result retrieved from 
the PT server identifies sequences using the \Arg{name} field. For completely different databases, this
will usually result in SINA being unable to find reference sequences. It may, however, also result
in SINA retrieving the wrong sequences. 
\item[\Opt{-{}-turn } \{\Arg{none}\Bar\Arg{revcomp}\Bar\Arg{all}\}]
Using this option, SINA can be configured to automatically reorient input sequences. If set to \Arg{none}, 
automatic reorientation is disabled. If set to \Arg{revcomp} only the reversed and complemented orientation
of the input sequences is considered. If set to \Arg{all} all four combinations of reversing and complementing
the sequence are considered. The default is \Arg{all}. Turning this feature off or reducing its scope 
will improve performance.

To determine which orientation is most likely, SINA uses the PT server to search for the sequence in the
configured orientations. If an orientation different to the original yields a higher scoring best match, 
the sequence is modified accordingly.
\item[\Opt{-{}-realign}]
Configures SINA not to copy alignment information from identical reference sequences or reference
sequence of which the input sequence is a substring. 

Normally, SINA will compare the input sequence with all reference sequences found via the PT server search. 
If the input sequence is a substring of any of the reference sequences, the alignment of the reference
sequence of which the input sequence is a substring will be directly transferred to the input sequence.

If the input sequence is found to be an exact match to a reference sequence, this will be noted in the field
\Arg{align\_log\_slv} with the string ``\texttt{copied alignment from identical template sequence}''. If the input
sequence is found to be a substring of a reference sequence, this will be noted with the string
``\texttt{copied alignment from (longer) template sequence}''. In both cases, the contents of the fields \Arg{acc}
and \Arg{start} will also be logged to identify the reference sequence. 

If suitable sequences for alignment copying are found, but \Arg{-{}-realign} is set, the sequences will be 
removed from the alignment reference. This will be noted in the log with the message 
``\texttt{sequences [acc list] containing exact candidate removed from family;}''.
\item[\Opt{-{}-overhang} \{\Arg{attach}\Bar\Arg{remove}\Bar\Arg{edge}\}]
If the reference sequences used for alignment do not cover the input sequence completely, e.g. because it
contains bases beyond the gene boundary, these bases cannot be aligned. This option configures how
SINA handles such unaligned bases at the end of the input sequences. If set to \Arg{attach}, the bases 
will be placed in consecutive columns outwards from the last aligned base, i.e. they will be ``attached''
to the outer most aligned base. If set to \Arg{remove}, these bases will be omitted from the output. If
set to \Arg{edge}, these bases will be placed in consecutive columns inwards from the first and last
alignment column, i.e. ``moved to the edge of the alignment''. The default is \Arg{attach}. 
\item[\Opt{-{}-lowercase} \{\Arg{none}\Bar\Arg{original}\Bar\Arg{unaligned}\}]
Use this option to configure which bases you wish to be in lower case in the output. The default
setting is \Arg{none}, which will output all bases in upper case. If set to \Arg{original}, the original
cases will not be modified. If set to \Arg{unaligned}, the case will be used to convey which bases
of the input sequences remained unaligned by setting aligned bases to upper case and unaligned bases 
to lower case in the output. Unaligned bases are either overhang (see \Arg{-{}-overhang} above)
or result from insertions which could not be found in any of the reference sequences. If large
insertions required shifting aligned bases (see \Arg{-{}-insertion} below), the shifted bases will 
also be considered unaligned and shown in lower case. 
\item[\Opt{-{}-insertion} \{\Arg{shift}\Bar\Arg{forbid}\Bar\Arg{remove}\}]
Since SINA aligns sequences to match a given fixed column reference alignment, insertions in the input
sequences may have occurred that cannot be accommodated by the reference alignment. While the only
correct way of dealing with this is certainly inserting further columns into the reference alignment
to create sufficient room, this may not always be feasible. 

The default setting is to \Arg{shift} the
bases surrounding such a large insertion aside as required. This is done by iteratively choosing the
nearest free column to the left or right until sufficient columns have been found. Each time bases
are encountered between the insertion and the free column, these bases are added to the insertion. The
main benefit of this naive approach is that the position and size of insertions that could not be
accommodated are known. The message ``\texttt{shifting bases to fit in N bases at pos X to Y}'' will be logged
each time an insertion of length N is attempted between positions $X$ and $Y$ with $Y-X<N$. The affected
bases can be marked as unaligned by exporting them in lower case letters using the \Arg{-{}-lowercase}
option described above. A summary giving the total number of shifted bases and the longest insertion
is also logged for each sequence.

The option \Arg{forbid} configures SINA to instead disallow insertions that will not fit the reference
alignment during the dynamic programming stage of sequence alignment. While this option constitutes
a loss of optimality of the alignment algorithm if the gap extension penalty (see \Opt{-{}-pen-gapext}
below) is different from the gap open penalty (see \Opt{-{}-pen-gap} below) it results in slightly less
damage to the alignment accuracy. 

The option \Arg{remove} configures SINA to omit bases from insertions as necessary to fit these
insertions into the alignment without moving surrounding aligned bases. This option should be handled
with care as the original sequence is altered. If the alignment is subjected to column masking or
column sampling (such as during tree reconstruction with bootstrapping), omitting bases is safe, as
these methods interpret the resulting MSA from a column perspective.

Which option is the most suitable should be carefully considered for each use case. Whenever possible, 
circumventing the necessity to handle insertions that do not fit into the alignment by simply
adding gap columns into the reference alignment is the preferred solution.
\item[\OptArg{-{}-filter }{filtername}]
Using this option it is possible to configure using statistical information on positional variability
during the alignment. ``Filter'' is a colloquial term used for ``sequence associated information'' or 
SAIs as used by ARB. Filters/SAIs of the type ``positional variability by parsimony''
(PVP) are eligible for use via this parameter. Please consult the SINA publication and the ARB 
documentation for more information.
\item[\OptArg{-{}-auto-filter-field }{fieldname}]
This option allows automatically selecting a PVP filter based on strings contained in an ARB database 
field. If the configured field contains a shared prefix over all selected reference sequences, the 
ARB database configured with \Opt{-{}-ptdb} is searched for a matching filter. A filter is considered matching
if the part of its name following the first colon is a itself a prefix of the shared prefix described above.
As an example, if \Arg{tax\_slv} is chosen and all reference sequences share the prefix 
``Bacteria;Proteobacteria;'' then the filter ``silva\_108:Bacteria'' will match. If \Opt{-{}-filter} is also
provided, the \Arg{filtername} must match the part of the filter name before the first colon. 
\item[\OptArg{-{}-auto-filter-threshold }{value}]
The term ``shared prefix of all reference sequence'' can be relaxed to ``longest prefix shared by \Arg{value} of 
the reference sequence'' using this parameter. (Default: 0.8)
\item[\OptArg{-{}-fs-min }{value}]
The minimum number of reference sequences that should be used. If less matches are returned by the 
kmer search, less sequences will be used. (Default: 40)
\item[\OptArg{-{}-fs-max }{value}]
The maximum number of reference sequences that should be used. (Default: 40)
\item[\OptArg{-{}-fs-msc }{value}]
The minimal kmer score reference sequences should have with the input sequence. At least as many sequences
as configured by \Arg{-{}-fs-min} will be used. Up to \Arg{-{}-fs-max} sequences will be used \textbf{if} they have a kmer score higher than configured by \Arg{-{}-fs-msc}.  (Default: 0.7)
\item[\OptArg{-{}-fs-msc-max }{value}]
Limits sequence selection to sequences having kmer score no higher than \Arg{value}. (Default: 2, that is, disabled)
\item[\Opt{-{}-fs-leave-query-out}]
Setting this option will remove the query sequence from the reference sequences based on its \Arg{name}. This is sensible for
evaluation in comparison to other tools where leave-query-out style evaluation can only be done by excluding the exact
query sequence from the reference. If the alignment must not be directly derived from any reference sequence, even if
the reference dataset contains redundant data, \Opt{-{}-realign} should be used.
\item[\OptArg{-{}-fs-req }{value}]
The minimum number of reference sequences that must be used. If less matches are returned by the 
kmer search, alignment is refused. The sequence will not be contained in the output. (Default: 1)
\item[\OptArg{-{}-fs-req-full }{value}]
The minimum number of full length sequences that should be included in the reference. The matches from 
the kmer search are parsed until, beyond the limits given by \Arg{-{}-fs-max} and \Arg{-{}-fs-msc}, at least
\Opt{value} such sequences have been found and added to the reference sequences. 
\item[\OptArg{-{}-fs-full-len }{value}]
The minimum number of bases constituting a full-length sequence. 
\item[\Opt{-{}-fs-kmer-no-fast}]
Disable the PT server fast search. The fast kmer search considers only kmers beginning with 'A'.
\item[\OptArg{-{}-fs-kmer-len }{k}]
Configures the length k of the kmers used for the kmer similarity search. 
\item[\OptArg{-{}-fs-kmer-mm }{value}]
Configures the number of mismatching bases a kmer may have to be considered matching.
\item[\Opt{-{}-fs-kmer-norel}]
Computes the kmer score using the length of the query sequence only. If not set, the kmer score
is computed as the number of shared kmeres between query and match candidate divided by the length 
of the shorter. 
\item[\OptArg{-{}-fs-min-len }{value}]
Minimal length sequences found via the kmer search must have to be considered for inclusion into 
the reference sequences. 
\item[\OptArg{-{}-fs-weight }{value}]
Factor with which the frequency at which a base occurs within the reference sequences will be
used to weight match and mismatch scores between the base and bases from the input sequence. 
\item[\OptArg{-{}-gene-start }{value}]
Position within the alignment corresponding to the first base of the aligned gene. 
\item[\OptArg{-{}-gene-stop }{value}]
Position within the alignment corresponding to the last base of the aligned gene.
\item[\OptArg{-{}-fs-cover-gene }{value}]
Minimum number of times the gene-start and gene-stop positions are at least touched by one of the
reference sequences. If the above rules did not result in sufficient such sequences, further sequences
covering the respective position are added until the condition is met. 
\item[\OptArg{-{}-match-score }{value}]
The match score used during the dynamic programming stage of partial order alignment (POA).
\item[\OptArg{-{}-mismatch-score }{value}]
The mismatch score used during the dynamic programming stage of partial order alignment (POA).
\item[\OptArg{-{}-pen-gap }{value}]
The gap open penalty used during the dynamic programming stage of partial order alignment (POA).
\item[\OptArg{-{}-pen-gapext }{value}]
The gap extension used during the dynamic programming stage of partial order alignment (POA).
\item[\Opt{-{}-debug-graph}]
Enables dumping of graph data in graphviz format suitable for processing with e.g. \Cmd{dot}{1}. 
For each aligned sequence, the DAG used as alignment 
template is dumped. Subsections of the dynamic programming graph/mesh, each covering the same fractions
as shown with \Arg{-{}-show-diff}, are also dumped. Please be aware that the output will be huge. 
\item[\Opt{-{}-use-subst-matrix}]
Experimental. Do not use!
\end{Description}
\subsection{Search and Classification Options}
\begin{Description}
\item[\OptArg{-{}-search-db }{filename}]
Configures the name of the ARB database which should be used for sequence search. Unless \Arg{-{}-search-all}
is also set, a PT server will be started for this database. The same rules as for \Arg{-{}-ptdb} apply. It
is permissible to use the same file as in \Arg{-{}-ptdb}. In this case, the database will be loaded only 
once. 
\item[\OptArg{-{}-search-port }{socket}]
Configures the port on which SINA should communicate with the PT server used for kmer searching. The
same rules as for \Arg{-{}-ptport} apply. If \Arg{-{}-search-all} is set, no PT server will be used and
this setting will be ignored.
\item[\Opt{-{}-search-all}]
Configures SINA to compare the aligned input sequence with \textbf{all} sequences contained in the 
database given by \Arg{-{}-search-db}. No PT server will be used. 
\item[\Opt{-{}-search-no-fast}]
Disable the PT server fast search. The fast kmer search considers only kmers beginning with 'A'.
\item[\OptArg{-{}-search-kmer-candidates }{n}]
Configures the number of best matching results from the kmer search that should be compared
with the input sequences based on the alignment. 
\item[\OptArg{-{}-search-kmer-len }{arg}]
Configures the length k of the kmers used for the kmer similarity search.
\item[\OptArg{-{}-search-kmer-mm }{arg}]
Configures the number of mismatching bases a kmer may have to be considered matching.
\item[\Opt{-{}-search-kmer-norel}]
Computes the kmer score using the length of the query sequence only. If not set, the kmer score
is computed as the number of shared kmeres between query and match candidate divided by the length 
of the shorter. 
\item[\OptArg{-{}-search-min-sim }{value}]
Minimal identity a sequence must have with the input sequence to be included in the search result.
\item[\Opt{-{}-search-ignore-super}]
Exclude sequences of which the input sequence is a substring from the search result.
\item[\OptArg{-{}-search-max-result }{value}]
Limit the maximum number of search results per input sequence.
\item[\OptArg{-{}-search-copy-fields }{fieldnames}]
Configures a colon separated list of ARB fields which will be copied into the input sequence. The field 
be prepended with ``copy\_$<$accession$>$\_'' in the output to indicate from which search result
the data came. 
\item[\OptArg{-{}-lca-fields }{fieldnames}]
Derives a LCA classification of the input sequence from the classifications of the sequences found in the search. This feature requires the reference database to contain a field specifying the sequence classifications in 
materialized path format (i.e. ``Bacteria;Proteobacteria;...''). The ``least common ancestor'' is the shared 
prefix of these strings. Prefixes must always end with a semicolon. Depending on the desired rank up to 
which the sequences should be classified, appropriate sequence similarity cutoffs should be configured with
\Arg{-{}-search-min-sim}. It is possible to specify multiple source taxonomies as \Opt{fieldnames} by passing colon
separated list. Derived LCA classification will be stored in fields named ``lca\_$<$fieldname$>$''. 
\item[\OptArg{-{}-lca-quorum }{value}]
Relaxes LCA classification from ``shared by \textbf{all} search results'' 
to a fraction \Opt{value} of the search results. 
\end{Description}
\section{Generated Meta Data Values}
\begin{Description}
\item[\Arg{align\_bp\_score\_slv}]
  This is a score calculated from the aligned sequence and the HELIX SAI. If the reference database
  contains no HELIX SAI the score will be NaN. Otherwise, the score is computed as follows. For each
  pair of columns covered by the aligned sequence a score of 1 is awarded if the pair is AU, GU or GC;
  a score of 0 is awarded if the pair is AG or GG; a score of -1 is awarded if the pair is AA, AC, CC,
  CU or UU or if one of the columns contains a gap character; the sum of these scores is divided by
  the number of considered columns. The value is scaled to match the range between 0 and 100.

  This value is likely to change or disappear in future versions.
\item[\Arg{align\_cutoff\_head\_slv}]
  This is the number of bases at the beginning of the sequences that remained unaligned.

\item[\Arg{align\_cutoff\_tail\_slv}]
  This is the number of bases at the end of the sequences that remained unaligned.

\item[\Arg{align\_family\_slv}]
  This is a list of the sequences that were used to build to align the input sequence.

\item[\Arg{align\_filter\_slv}]
  If a PVP filter was applied, the name of that filter will be stored in this field. 
\item[\Arg{align\_log\_slv}]
  Messages generated during the alignment process will be logged here. 
\item[\Arg{align\_startpos\_slv}]
  This is the alignment position (column number) of the first aligned base.
\item[\Arg{align\_stoppos\_slv}]
  This is the alignment position (column number) of the last aligned base.
\item[\Arg{aligned\_slv}]
  This is the current date.
\item[\Arg{full\_name}]
  If FASTA is chosen as input format, this field will contain the part of the FASTA
  header lines after the first space character.
\item[\Arg{nearest\_slv}]
  This field contains a space separated list of the results from the homology 
 search stage. Each search result is given in the following form:
 ``\texttt{<accession>.<version>:<start>:<stop>~<identity>}''
\item[\Arg{nuc}]
  The number of nucleotides in the input sequence.
\item[\Arg{nuc\_gene\_slv}]
  The number of nucleotides in the sequence aligned to be within the gene borders.
\item[\Arg{turn\_slv}]
  Documents actions taken by the automatic reorientation of sequences. Possible values
  are ``disabled'', ``none'', ``reversed'', ``complemented'' and 
  ``reversed and complemented''.
\item[\Arg{sps}]
  This field contains the fractional identity of the aligned input sequence with 
  the input sequence in its original alignment. The number of identical base/column
  pairs is divided by the number of nucleotides. 
\item[\Arg{error}]
  The number of differing base/column pairs divided by the number of nucleotides.
  The sum of \Arg{error} and \Arg{sps} may be larger than 1 because of gap characters.
  If in the new alignment, a base ends up in what should be a gap position and a gap
  is placed where the base was in the original alignment, two misaligned positions
  are found.
\item[\Arg{matches}]
  Number of identical base/column pairs in SINA aligned sequence and input alignment.
\item[\Arg{mismatches}]
  Number of differing base/column pairs in SINA aligned sequence and input alignment.
\item[\Arg{bps}]
  The same as \Arg{slv\_bp\_score} but unscaled and not rounded to integer.
\item[\Arg{cpm}]
  Correctly placed mutations, or rather, an attempt at calculating such a measure
  intended to be used as a measure of alignment accuracy independent of the 
  identity an input sequence as with its closest reference sequences. 
  The value is the number of base/column pairs the aligned sequence
  shares with its original alignment \textbf{more} than the sequence in its 
  original alignment shares with the closest found reference sequence divided by
  the number of base/column pairs in original alignment that are not matched 
  by the closest reference.

  This value is likely disappear or change in future versions. 
\item[\Arg{idty}]
  The highest fractional identity of the input sequence with any of the 
  selected reference sequences calculated as the number of matches (see above)
  divided by the length of the input sequence.
\item[\Arg{achieved\_idty}]
  Identical to \Arg{idty} but using the SINA alignment rather than the original
  alignment. 
\item[\Arg{lca\_*}]
  These fields contain the classifications derived via LCA.
\item[\Arg{copy\_*}]
  These fields contain the data copied from the search results.
\end{Description}
\section{Examples}
\begin{Description}
\item[Aligning some sequences]
To align sequences, you need to get a suitable reference alignment in ARB 
format. If you have LSU or SSU sequences to align, the Ref or RefNR 
datasets from \URL{www.arb-silva.de} work well. Otherwise, check
below for an example on how to convert your own multi-fasta reference
alignment to ARB format. 

\begin{verbatim}
 ./sina -i mysequences.fasta -o alignedsequences.fasta \
        --ptdb reference.arb
\end{verbatim}

The first time you run this, a PT server will be started and will 
begin building its index. The index is stored in 
\texttt{reference.arb.pt} and will only be computed again if 
reference.arb changes (the decision is made based on file timestamps
only). The PT server will also continue to run once it has
been started. Subsequent sina runs will be much faster therefore.
Nonetheless, start-up time may be long if \texttt{reference.arb}
is large. 

\item[Classifying some sequences]
If you are using a reference database that has a field containing 
classifications, you can use SINA to classify your sequences. The
SILVA Ref database contain several taxonomies in fields beginning 
with ``\texttt{tax\_}''. To classify sequences based on 
the SILVA taxonomy, you can use this command line:

\begin{verbatim}
 ./sina -i mysequences.fasta -o alignedsequences.fasta \
        --meta-fmt csv \
        --ptdb reference.arb \
        --search --search-db reference.arb --lca-fields tax_slv
\end{verbatim}

The classifications will be (among the other values) written to 
alignedsequences.fasta.csv to the column labeled 
``\texttt{lca\_tax\_slv}''. 
\item[Converting FASTA to ARB]
By disabling all stages, SINA can be used to convert between ARB and 
FASTA format (in a limited fashion, use ARB if you want to do more
fancy stuff):

\begin{verbatim}
 ./sina -i mysequences.fasta -o mysequences.arb --prealigned
\end{verbatim}

This will generate an ARB file from your aligned sequences suitable 
for use as a reference MSA. The first word of each FASTA header 
will be written to the ARB field ``\texttt{name}''. Make sure they 
are unique for each sequence. ARB uses this field to identify 
sequences, duplicates will overwrite the previous sequence with 
the same name. The remainder of the fasta header will be written
to the field ``\texttt{full\_name}''. 

\item[Running a leave-query-out accuracy benchmark]
You can run a quick check on the accuracy achieved by SINA with 
your reference MSA by having it align each of those sequences 
(ignoring the same sequence in the process) and log the 
accuracy with which it could reproduce the original alignment.

\begin{verbatim}
 ./sina -i myreference.arb --ptdb myreference.arb \
        -o /dev/null --outtype fasta \
        --fs-leave-query-out --show-dist
\end{verbatim}

The average accuracy will be printed at the end of the SINA run.
\item[Converting FASTA output from RNA to DNA]
SINA writes IUPAC encoded RNA as its output. If you require
DNA, you can simply convert the file using this command:

\begin{verbatim}
sed '/^[^>]/ y/uU/tT/' rna.fasta > dna.fasta
\end{verbatim}
\end{Description}
%\section{Files}
\section{See Also}
ARB, \URL{http://www.arb-home.de}\\
SILVA, \URL{http://www.arb-silva.de}
%\section{Requirements}
\section{Version}
Version: \Version\ of \Date.
\section{License and Copyright}
\begin{Description}
\item[Copyright \copyright\ 2006-2011] Elmar Pruesse (\Email{epruesse@mpi-bremen.de})
\item[License] This copy of SINA is licensed under the SINA PUEL (see below).

The author of SINA reserves all copyrights and other intellectual property rights. All further rights are at Ribocon GmbH (the "Owner") in legal agreement with the author of SINA and all third parties involved.

If you are interested in commercial use of the SILVA stand-alone software contact \Email{sina@ribocon.com}.

\textbf{Personal Use and Evaluation License (PUEL) for SINA Stand-Alone Software}

This license applies if you download the SINA Stand-Alone Software Package (the "Product") from \URL{www.arb-silva.de}. In summary, the license allows you to use the Product free of charge for academic Personal Use or, alternatively, for non-academic, time-limited Evaluation. 

\textbf{Overview:}
Personal Use (academic) is when you install the Product yourself and you make use of it. You can use the Product within an academic study to process as much data as you like and publish the processed data as long as you follow the terms below. If you deploy the Product to a single or multiple computers for colleagues within your institution, e.g. in the capacity as a system administrator, this would no longer qualify as Personal Use.

Personal Use does NOT include (1) any redistribution of the Product, (2) any kind of Product-based data analysis service for third parties, or (3) integration of the Product into another software. 

\textbf{License Agreement:}
You should have received a copy of the license agreement with this software in the file LICENSE.txt. If you did not, please
visit \URL{http://www.arb-silva.de/aligner/sina}.

\end{Description}
\LatexManEnd
\end{document}
